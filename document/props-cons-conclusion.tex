\section{Kelebihan dan Kekurangan}
Permasalahan utama dalam klasifikasi adalah sulitnya memngklasifikasikan data non-linear. 
Salah satu contoh masalahnya adalah klasifikasi gerbang XOR. 
\textit{Neural Network} dapat menyelesaikan masalah klasifikasi non-linear tersebut karena \textit{Neural Network} menggunakan bantuan layer tambahan dalam proses klasifikasinya.
\textit{Neural Network} memiliki sifat generik, artinya untuk satu algoritma dapat menyelesakan masalah untuk beberapa jenis training data.

Salah satu persoalan utama \textit{Neural Network} adalah sulitnya implementasi dan interpretasi. 
Namun saat ini sudah banyak perangkat lunak yang menyediakan \textit{built-in} untuk menyelesaikan masalah dengan \textit{Neural Network}.
Dalam penggunaan \textit{Neural Network} ditemukan masalah sulitnya menentukan arsitektur dari jaringan, seperti menentukan banyaknya layer, banyaknya node pada layer.
Pemilihan banyak node pada layer yang terlalu sedikit akan mengakibatkan \textit{backpagation} tidak mendekati nilai minimum saat training.
Terlalu banyak node pada layer dapat menyebabkan \textit{over fitting} hasil training data dan mengakibatkan hasil generalisasi yang buruk.
Masalah terbesar dalam \textit{Neural Network} adalah kompleksitas waktu yang sangat tinggi. 
Kompleksitas \textit{backpropagation} tinggi dikarenakan \textit{curse of dimentionality}, jika dimensi data bertmbah, maka jumlah data training yang diperlukan juga signifikan bertambah.


\section{Kesimpulan}

	\subsection{Ringkasan}
	\lipsum

	\subsection{Arahan Kedapan}
	\lipsum[1]
