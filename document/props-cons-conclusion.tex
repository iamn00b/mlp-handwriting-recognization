
\section{Kelebihan dan Kekurangan}
\lipsum

\section{Kesimpulan}

	\subsection{Ringkasan}
	\lipsum[1]

	\subsection{Arahan Kedapan}
	\lipsum[1]


pros 
	dapat memecahkan pasalah yang kompleks, nonlinear
	dapat beradaptasi dengan beberapa jenis training data

cons 
sulit diimplementasi dan diinterpretasi
	tersedia software yang menyediakan built-in solusi.
sulit menentukan aritektur dari jaringan
	banyaknya layer
	banyaknya node pada layer
		tergantung kepada kompleksitas dari input dan output mapping. 
		terlalu sedikit backpropagation algorithm gagal menuju konvegen minimun saat training 
		terlalu banyak overfitting hasil training data dan performa generalisasi buruk
kompleksitas backpropagation tinggi (curse of dimentionality)
	jika dimensi data bertambah, maka jumlah data training yang diperlukan juga signifikan bertambah


	% $$http://www.researchgate.net/profile/Stephen_Dorling/publication/263416087_Artificial_neural_networks_(the_multilayer_perceptron)-a_review_of_applications_in_the_atmospheric_sciences_-_application_for_wind_retrieval_from_spaceborne_scatterometer_data/links/00b495364b91edd43a000000.pdf$

